\section{Programmazione multi obbiettivo}
Estensione della programmazione lineare. Il problema di ottimizzazione per un problema a più obbiettivi è diviso in due parti:

\begin{enumerate}
	\item calcolo della regione Pareto-ottima,
	\item scelta di una soluzione ottima tra quelle individuate nel passo precedente.
\end{enumerate}

In questo si considera solo la programmazione a due obbiettivi. Le soluzioni di un problema a più obbiettivi possono essere rappresentate nello spazio delle soluzioni $f_1(x) \times \dotsc f_k(x)$.

\subsection{Dominanza}
Dati $f_1(x), \dotsc, f_k(x)$ funzioni obbiettivo e due soluzioni ammissibili $x^{'}, x^{''}$, $x^{'}$ domina $x^{''}$ se e solo se:

\begin{equation}\label{dominanza}
	\begin{cases}
		f_i(x^{'}) \leq f_i(x^{''}) \quad \forall i \in 1, \dotsc, k \\
		\exists j \in 1, \dotsc, k : f_j(x^{'}) < f_j(x^{''})
	\end{cases}
\end{equation}

Un interpretazione geometrica della dominanza in due dimensioni consiste nell'immaginare un rettangolo con un vertice nell'origne e l'altro nel punto $x^{''}$. Se il punto $x^{'}$ è all'interno del rettangolo allora la soluzione $x^{''}$ domina $x^{'}$.

\subsection{Regione Pareto ottima}
In programmazione a più obbiettivi il concetto di soluzione ottima è sostituito da quello di dominanza. L'insieme di tutte le soluzioni dominate definisce la regione Pareto ottima. Esistono due modalità per detrrminare la regione Pareto-ottima:

\begin{itemize}
	\item metodo dei pesi,
	\item metodo dei vincoli.
\end{itemize}

\subsubsection{Metodo dei pesi}
Questo metodo combina linearmente le funzioni obbiettivo in un unica attraverso dei coefficienti $\lambda$ tali che:

\begin{equation*}
	\lambda_i \geq 0 \quad \text{e} \quad \sum_k \lambda_k = 1
\end{equation*}

A questo punto si può calcolare regione Paretiana esegundo l'analisi parametrica su $\lambda$. 

\subsubsection{Metodo dei vincoli}
Invece il metodo dei vincoli ottimizza una sola funzione obbiettivo trasformando le altre in vincoli con un termine noto parametrico $\beta$. Per trovare la regione di Pareto devo eseguire l'analisi parametrica su $\beta$.

\paragraph{Regione paretiana discreta}
Il metodo dei pesi in generale non garantisce di trovare tutte le soluzioni paretiane, quindi per effettuare l'analisi della regione di Pareto si utilizza il metodo dei vincoli.

\subsection{Scelta soluzione ottima}
La seconda fase è supportata da metodi quantitativi, ovvero richiede una scelta che però non è demandabile ad un algoritmo.
Esistono diversi metodi per questo scopo:

\begin{itemize}
	\item metodo delle curve di indifferenza,
	\item criterio di massima curvatura,
	\item criterio del punto utopia,
	\item criterio degli standard.
\end{itemize}
