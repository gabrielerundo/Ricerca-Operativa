\chapter{Programmazione lineare}

Un problema è di programmazione lineare se:

\begin{itemize}
    \item le variabili hanno un dominio continuo,
    \item i vincoli sono equazioni e disequazioni lineari delle variabili,
    \item la funzione obbiettivo è una funzione lineare delle variabili.
\end{itemize}

\section{Forma generale}
%Introdurre acronimo di PL
Forma del problema di PL con una funzione obbiettivo lineare rispetto alle variabili, vincoli in forma di equazioni o disequazioni e delle variabili libere. 

\section{Forma alle disuguaglianze}
Rappresentazione del problema di PL dove non sono presenti vincoli in forma di equazioni, variabili libere e tutti i vincoli sono di verso opposto a quello della funzione obbiettivo.
Per ottenere tale forma è necessario:
\begin{itemize}
	\item eliminare i vincoli di uguaglianza attraverso sotituzioni,
	\item eliminare variabili libere attraverso sottrazione tra variabili non negative.
\end{itemize}

%scrivere una formula per rappresentare forma disuguaglianze
\begin{equation} \label{esFormaDis}
	\begin{split}
		\text{max } w& = +x_3 +3x_4 -3x_5\\
		\text{s.t. } & \quad -3x_3 +2x_4 -2x_5 \le 9 \\
		& \quad  -4x_3 -3x_4 +3x_5 \le 8\\
		& \quad x_3, x_4, x_5 \ge 0
	\end{split}
\end{equation}

Questa forma offre un interpretazione geometrica del problema di PL.
Un vincolo rappresenta un semipiano convesso, la funzione obbiettivo corrisponde ad un fascio di iperpiani paralleli infine la direzione di ottimizzazione è definita da quale tipo di ottimizzazione è richiesta \textit{massimizza} o \textit{minimizza}.

%aggiungere una nota a pie pagina con la spiegazione per cui l'intersezione tra vincoli è un poliedro.
Il sistema di vincoli corrisponde ad un intersezione tra semispazi convessi. L'intersezione tra semispazi è un poliedro che può essere: 
\begin{itemize}
    \item limitato: almeno un vertice del poliedro corrisponde alla soluzione ottima,
    \item illimitato: non esiste valore ottimo se illimitato nella direzione di ottimizzazione,
    \item nullo: non ammette soluzioni.
\end{itemize}

\section{Forma standard}
In questa forma la funzione obbiettivo è messa in forma di minimo e tutti vincoli devono essere di uguaglianza, attraverso l'introduzione di variabili di \textit{slack} e \textit{surplus}.

\begin{equation} \label{formaStandard}
	z=\min \{c^Tx:Ax=b, x\geq0\}
\end{equation}

\subsection{Variabili di slack e surplus}
Sono variabili non negative che vengono introdotte per trasformare i vincoli dalla forma di disequazione a quella di equazione.
Per realizzare tale trasformazione devo:

\begin{itemize}
    \item Sommo variabili di \textit{slack} quando il segno del vincolo è $\leq$,
    \item Sottraggo variabili di \textit{surplus} quando il segno del vincolo è di $\geq$.
\end{itemize}

\subsection{Soluzione di base}

Dopo aver introdotto le variabili di slack e surplus il sistema dei vincoli è formato da $m$ equazioni lineari in $n+m$ variabili. 
Il sistema per avere soluzione unica deve eliminare gli $n$ gradi di libertà in eccesso. Ogni variabile nulla nella forma standard corrisponde un vincolo attivo nella forma alle disuguaglianze.

Una base è un sottoinsieme di $m$ variabili tra le $n+m$. La scelta di una base comporta in $A=[B|N]$. Quindi la scelta di una base si può scrivere in questo modo:
%imparare a scrivere più formule nello stesso blocco equation, come se fossero dei passaggi
\begin{equation}\label{sceltaBase}
    \begin{array}{c}
    Bx_b+Nx_n=b \\
    B^{-1}Bx_b+B^{-1}Nx_n=B^{-1}b \\
    x_b=B^{-1}b-B^{-1}Nx_n
    \end{array}
\end{equation}

\section{Teorema fondamentale di PL}
Dato un problema di PL in forma standard \ref{formaStandard}, $z=\{c^Tx:Ax=b, x\geq0\}$ con A di rango $m$:
\begin{itemize}
    \item se esiste una soluzione ammissibile allora esiste anche una soluzione ammissibile di base,
    \item se esiste una soluzione ottima allora esiste anche una soluzione ottima di base.
\end{itemize}
Questo teorema permette di trasformare un problema nel continuo in uno nel discreto cercando soluzioni di base. 
Il metodo risolutivo più diffuso è l'algoritmo del simplesso.
