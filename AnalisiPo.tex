\section{Analisi post-ottimale}
Dopo aver trovato una soluzione ottima, voglio studiare quanto tale soluzione sia robusta, ovvero quanto la scelta di una base non cambia rispetto a una piccola variazione dei dati.

I dati che possono variare sono:

\begin{itemize}
	\item coefficienti $a_{ij}$ difficili da trattare,
	\item coefficienti $b_j$, $c_i$ oggetto delle nostre analisi.
\end{itemize}

\subsection{Analisi di sensitività o stabilità}
Lo scopo di questa analisi è determinare l'intervallo per cui può variare $b_j$ o $c_i$ senza che cambi la base. Ovvero devo mantenere ottimalità e ammissibilità.

\begin{itemize}
	\item ammisibilità, $B^{-1}b \geq 0$, che dipende solo da b,
	\item ottimalità, $c_N - B^{-1}Nc_B \geq 0$, che dipende solo da c.
\end{itemize}

\subsubsection{Variazione coefficiente $c_j$}
Il coefficiente $c_j$, se si osserva geometricamente il problema, influisce sul coefficiente angolare della funzione obbiettivo. 

L'obbiettivo di questa analisi è quindi quello di trovare di quanto posso inclinare la retta senza che cambi la base. Ovvero cambiare l'inclinazione della funzione obbiettivo finché non diventa parallela al vincolo.

%immagine o pacchetto con TikZ per rapprensentare il la funzione obbiettivo che cambia inclinazione

Tutti i dati necessari per l'analisi di sensitività sono contenuti nel tableau all'ottimo che si distinguono da quelli iniziali per l'uso di *.
Supponiamo di avere un problema nella forma delle disuguaglianze con una funzione obbiettivo da massimizzare e tutti i vincoli di disuguaglianza $\leq$, si consideri la colonna $\bar{j}$ la formula di $\triangle c_{\bar{j}}$ sono possibili due casi: 

\paragraph{La colonna scelta è in base $\bar{j} \in B$}

\begin{equation} \label{variazioneC}
	\max \lbrace- \infty, \max_{j \in N} \lbrace \frac{-c_j^*}{a_{\bar{r}j}^{*+}} \rbrace \rbrace \leq \triangle c_{\bar{j}} \leq \min \lbrace \min_{j \in N} \lbrace \frac{-c_j^*}{a_{\bar{r}j}^{*-}} \rbrace,+ \infty \rbrace
\end{equation}

\paragraph{La colonna scelta è fuori base $\bar{j} \in N$}

\begin{equation}
	\triangle c_{\bar{j}} \leq c_{\bar{j}}^*
\end{equation}

%le parentesi graffe sono piccole, come faccio a rendere più grandi?

%spiegare il significato degli elementi della formula

\subsubsection{Variazione coefficiente $b_i$}
Il coefficiente $b_i$, se si osserva geometricamente il problema, corrisponde all'intercetta di un vincolo del problema. 

Quindi per trovare l'intervallo in cui varia $b_i$ devo traslare il vincolo selezionato finché il suo stato non cambia \footnote{Con stato si intende il passaggio da attivo a non attivo o viceversa.}.

Anche in questo caso, scelta una riga $\bar{i}$, la formula per $\triangle b_{\bar{i}}$ varia a seconda di due casi:

\paragraph{Il vincolo $\bar{i}$ è attivo}

\begin{equation} \label{variazioneB}
	\max \lbrace- \infty, \max_i \lbrace \frac{-b_i^*}{a_{\bar{r}j}^{*+}} \rbrace \rbrace \leq \triangle b_{\bar{i}} \leq \min \lbrace \min_i \lbrace \frac{-b_i^*}{a_{\bar{r}j}^{*-}} \rbrace,+ \infty \rbrace
\end{equation}

\paragraph{Il vincolo $\bar{i}$ non è attivo}

\begin{equation}
	\triangle b_{\bar{i}} \geq -x_{\bar{j}}^*
\end{equation}

\subsection{Analisi parametrica}
Studia come $z*$ dipende dal valore del termine noto di un vincolo $b_i$.
Quest'analisi produce una funzione lineare a tratti dove ogni punto di discontinuità corrisponde ad un cambio di base mentre un segmento corrisponde ad una base ottima.

Per trovare il segmento corrispondente a una base ottima devo fare un analisi di sensitività su $b_i$. Ripetendo questo procedimento per tutti i segmenti mi permette di tracciare il grafico di $z^*$ al variare di $b_i$.

\subsection{Interpretazione economica della PL}
Questa interpretazione è strettamente legato al bilanciamento della produzione, per esempio il mix produttivo ottimale.
Dato il vettore dei coefficienti di costo ridotto $c^T = [ c_1 \dotsm c_n c_{n+1} \dotsm c_{m+n} ]$ i primi $n$ rappresentano i coefficienti di costo ridotto delle variabili, mentre gli altri $m$ sono i c.c.r. delle variabili dei vincoli detti anche prezzi ombra.

All'ottimo gli shodow price rappresentano il massimo prezzo a cui comprare la risorsa o il minimo prezzo a cui vendere le risorse extra. 
Il coefficiente di costo ridotto di variabili in base è nullo perché il ricavo marginale e il costo marginale risultano uguali.

Il costo ridotto $\bar{c}_j$ di ogni variabile $x_j$ è dato da:

\begin{equation} \label{costoRidotto}
	\bar{c}_j = c_j - \sum_i a_{ij} \lambda_i
\end{equation}
