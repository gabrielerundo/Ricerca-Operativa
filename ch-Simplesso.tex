\chapter{Algoritmo del simplesso}
Partendo da una soluzione ammissibile l'algoritmo del simplesso, con le sue iterazioni, mantiene l'ammissibilità e cerca l'ottimalità.

\section{Forma canonica}
Partendo dalla forma stamdard, questa forma corrisponde alla scelta di una base di solito si sceglie la base canonica $I$.
Estraendo la base $B$ da $A$ secondo formula (\ref{sceltaBase}) e poiché $B$ è invertibile, posso ricavare $x_b$:

\begin{equation}\label{xB}
x_b=B^{-1}b-(B^{-1}N)x_n
\end{equation}

Sostituendo $x_b$ nella funzione obbiettivo permette di ottenere:

\begin{equation}\label{formaCanonicaEstesa}
    \begin{array}{c}
	\min z=c_b^TB^{-1}b+(c_n^T-c_b^TB^{-1}N)x_n \\
	x_b+(B^{-1}N)x_n=B^{-1}b \\
	x_b,x_n\geq0
    \end{array}
\end{equation}

È possibile riscrivere il problema in forma più compatta nel seguente modo:

\begin{equation}\label{formaCanonicaCompatta}
    \begin{array}{c}
		\min z=z_b+\bar{c}_n^Tx_n \\
		\text{subject to } x_b+(B^{-1}N)x_n=B^{-1}b \\
		x_b,x_n\geq0
    \end{array}
\end{equation}

\subsection{Forma canonica debole}
\begin{itemize}
	\item le variabili in base non sono presenti nella funzione obbiettivo,
	\item I coefficienti delle variabili in base formano una matrice identità $I\in M(m,m)$.
\end{itemize}

Se queste condizioni sono verificate allora il problema è in forma canonica debole, ovvero una soluzione inamissibile.

Secondo l'interpretazione geometrica una forma non ammissibile è un intersezione dei vincoli fuori dal poliedro.

%Introdurre una figura per rappresentare una soluzione inammsissibile.

\subsection{Forma canonica forte}
Se oltre alle precedenti condizioni vale anche la seguente:
\begin{itemize}
	\item i termini noti dei coeficienti dei vincoli sono non negativi, ovvero $\bar{b}\geq\bar{0}$.
\end{itemize}

Allora il problema di PL è in forma canonica forte che corrisponde ad una soluzione ammissibile del problema. Secondo l'iterpretazione geometrica questa corrisponde ad un vertice del poliedro.

%pacchetto tick per disegnare su latex

\section{Il tableau}
Il tableau è una modalità di rappresentazione del problema di PL e la struttura dati su cui opera l'algoritmo.

\begin{equation}\label{tableau}
	\begin{array}{c|cc}
		-z_b&\bar{c}_N^T&0\\
		\hline
		\bar{b}&\bar{N}&I	 
	\end{array}
\end{equation}

Il significato dei valori:
\begin{itemize}
	\item $-z_b$ : valore della funzione obbiettivo cambiato di segno,
	\item $\bar{c}_N^T$ : vettore dei coefficienti  di costo ridotto,
	\item $\bar{b}$ : vettore dei termini noti dei vincoli.
\end {itemize}

%Traduzione da tableau a sistema di equazioni

\paragraph{Test ottimalità}
Osservando i coefficienti di costo ridotto $\bar{c}_N$ è possibile capire se un problema è ottimo, poiché rappresentano di quanto la funzione obbiettivo aumenterebbe se la variabile fuori base ad esso associata entrasse in base.

Quando $\bar{c}_N>0$ e il problema è ammissibile allora la soluzione è ottima.

\paragraph{Test di illimitatezza}
Osservando i valori di una colonna candidata per eseguire un passo di pivot. Se tutti i suoi temrini sono negativi, ovvero $a_{i^*j} < 0$, allora il problema associato è illimitato.

\section{Pseudocodice dell'algoritmo}
L'algoritmo è diviso in due fasi:
\begin{enumerate}
 	\item ricerca dell'ammissibilità, analizzando i coefficienti $\bar{c}_N$, 
	\item ricerca dell'ottimalità, analizzando il vettore dei termini noti $\bar{b}$.
\end{enumerate}
	
Ogni iterazione dell'algoritmo corrisponde ad un passo detto \textit{pivot}.

\section{Pivot}
Il passo di pivot corrisponde ad un cambio di base, ovvero una variabile in base esce ed una fuori entra.\footnote{In particolare entra in base la variabile il cui indice corrisponde alla colonna del pivot ed esce quella corrispondente alla riga del pivot.} 

Secondo l'interpretazione geometrica il passo di pivot corrisponde al passaggio di un vertice ad un'altro del poligono.

Il passo di pivot consiste nell'esecuzione di queste operazioni:
\begin{enumerate}
	\item scelgo un elemento positivo $a_{r,c}$ su una colonna con coefficiente di costo ridotto negativo, 
	\item divido la riga $r$ per il pivot, 
	\item sottraggo ad ogni riga $i\neq r$ la riga $r$ per $a_{i,c}$.
\end{enumerate}

Devo introdurre delle regole sulla scelta delle righe e delle colonne per applicare il passo di pivot sull'elemento \textit{corretto} così da mantenere l'ammissibilità e cercare l'ottimalità.

\subsection{Regola di scelta della colonna}
Per garantire che l'algoritmo sia privo di cicli infinti e ricerchi l'ottimalità devo scegliere la colonna con costo ridotto negativo. 
Nel caso in cui le colonne possibili siano più di una devo scegliere una tra queste opzioni:
\begin{itemize}
	\item scelgo la colonna con il costo ridotto negativo minore,
	\item ordino le colonne e prendo la prima colonna negativa,
	\item scelgo la colonna con il maggiore miglioramento della funzione obbiettivo,
	\item scelgo una colonna negativa a caso. 
	%$|\frac{a_{i,0} \cdot a_{0,j}}{a_{i,j}}|$  
\end{itemize}

\subsection{Regola di scelta della riga}
Per garantire invece che l'algoritmo continui a considerare una soluzione ammissibile della soluzione devo scegliere sulla colonna del pivot un elemento su una riga che:
\begin{itemize}
	\item valore positivo,
	\item minimo rapporto tra termine noto $b_i$ e candidato pivot $a_{ij}$.
\end{itemize}

Il significato geometrico di tale scelta corrisponde a muoversi, all'interno del poliedro, fino ad un'altra soluzione di base.

\section{Inizializzazione}

In alcuni casi è possibile che la base di partenza sia non ammissibile, e per applicare l'algoritmo del simplesso devo partire da una base ammisibile. 

Ci sono tre metodi per passare da una base non ammisibile ad una ammissibile:

\begin{enumerate}
	\item metodo delle variabili artificali,
	\item metodo big M,
	\item metodo Balinski-Gomori.
\end{enumerate}

\subsection{Metodo delle variabili artificiali}
Dato un modello di PL in forma standard \ref{formaStandard} si introduce una variabile artificiale $u_i\geq0$ con coefficiente $1$ per ogni vincolo definendo così un problema artificiale:

%introdurre formula del problema artificiale
\begin{equation} \label{problemaArtificiale}
z^{a} = \min \{ e^{T}u : Ax+Iu=\bar{b} \}
\end{equation}

Questo problema artificale non soddisfa le condizioni di ammisibilità. Devo scrivere le $u_i$ come combinazione lineare delle $a_{ij}$ affinché non compaiano nella funzione obbiettivo.

Se il valore all'ottimo del problema artificiale è nullo, $z^a=0$, allora questa soluzione del problema ausiliario corrisponde ad una soluzione ammissibile nel problema originale, altrimenti il problema orifinario è inamissibile.
%per quale ragione? 

Utilizzando questo approccio il tableau diventa più complesso, tuttavia è possibile semplificare il problema artificiale introducendo variabili artificali sono su alcuni vincoli.

\begin{itemize}
	\item Per i vincoli a cui è associata una variabile di slack non è necessario associare una variabile artificale, %perché?
	\item mentre per quelli a cui è associata una varibile di surplus ad ogni vincolo. %Questa parte non è molto chiara
\end{itemize}

\subsection{Metodo big M}
Invece di elimanre le variabili $x$ dal problema artificiale le penalizzo utilizzando coefficienti $M$ molto grandi.
Si risolve quindi un problema dove:

\begin{equation} \label{MetodoBigM}
\min z = \{ c^{T}x+w^{T}u=b , x,u \in \Re^{n}_+ \} \text{	dove } w^{T} = [M, \dotsc, M] 
\end{equation}

Utilizzando questo approccio un problema associato è la scelta di $M$.

\subsection{Metodo Balinski Gomory}
Questo approccio mantiene sempre le stesse dimensioni del tableu originale, accettando di lavorare su una forma canonica debole eseguendo passi di pivot sui vincoli violati, ovvero quelli con $b_i < 0$.

Il procedimento funziona in questo modo:
\begin{itemize}
	\item considero un vincolo violato,
	\item lo considero come obbiettivo e cerco di ottimizzarlo, sostitendolo con la funzione obbiettivo senza però selezionando gli elementi della riga come pivot,
	\item effettuo passi di pivot sul problema ausiliario.
\end{itemize}

Posso verificarsi diverse situazioni sul problema asiliario:

\begin{itemize}
	\item il problema ausiliario è inamissibile. 
	\item il problema ausiliario è illimiato.
	\item il problema ausiliario è limitato.
\end{itemize}

\subsubsection{Esempio di applicazione del metodo}
\[  
	\begin{array}{r|rrrrr}
		 0 &  0 & -1 & 0 & 0 & 0 \\
		\hline
		-2 &  1 & -2 & 1 & 0 & 0 \\
		-4 & -2 &  1 & 0 & 1 & 0 \\
		 4 &  1 &  1 & 0 & 0 & 1 
	\end{array}
\]

Questo tableau rappresenta un problema inammsibile, applico il metodo di Balinsk-Gomory e scelgo come obbiettivo un vincolo violato su cui effettuare il passo di pivot.

\[  
	\begin{array}{r|rrrrr}
		-4 & -2 &  1 & 0 & 1 & 0 \\
		\hline
		-2 &  1 & -2 & 1 & 0 & 0 \\
		 4 &  1 &  1 & 0 & 0 & 1 \\
		\hline
		 0 &  0 & -1 & 0 & 0 & 0 \\
	\end{array}
\]

A questo punto ottengo un problema ausiliario su eseguo diversi passi di pivot. 
